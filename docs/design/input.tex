\section{Proposed Input Specification}
\labelsection{input}

Below is a proposed input specification for MAM. The spec uses
[YAML](https://yaml.org/). Compared to Fortran namelists, YAML offers

\begin{itemize}
  \item {\bf better readability}: YAML files are flexible, and don't have a lot
    of "kibble" (braces, tags, and other stuff you see routinely in fancier
    markup like XML, HTML, and so on). They're very easy to read.
  \item {\bf better validation}: YAML files have named entries that can be
    used or ignored, depending on the needs of the particular application.
    Also, the file format allows for error checking.
  \item {\bf better language support}: Fortran namelists are so-called because
    they are only available to Fortran. YAML has libraries that allows it to
    be used with several programming languages (including Fortran!). This makes
    it easier for tools and other applications to use similar input files. In
    particular, this allows workflows to use a single input file to define a
    workload that can be processed with several tools.
  \item {\bf support for lists and maps}: YAML offers constructions for
    dynamically-sized datasets. As we move MAM toward runtime configurability,
    the ability to add species (and perhaps even modes) to an input file
    becomes more important.
  \item {\bf a larger support community}: Fortran remains a niche language,
    which often instills a sense of pride in scientists. Unfortunately, the
    downside to belonging to a small community of specialists is that the tools
    are invariably of lower quality than more commonly-used tools. One needs
    only to witness the notorious ongoing issues with Fortran compilers to see
    this phenomenon firsthand.
\end{itemize}

The YAML spec can be used both for initializing data via the API, and with
standalone drivers. We also provide a proposed API for reading data from
input files.

\subsection*{Sections}

A YAML file consists of several named sections. Each of these sections can
contain data and metadata. Sections are a powerful tool for organizing input
using simple concepts with human-readable notation.

\subsubsection*{Modes}

The \verb modes  section defines the particle size modes available to a MAM
model. As we discussed in \refsection{modes_and_species}, a mode has metadata
specifying its size range and its geometric standard deviation.

\begin{verbatim}
modes:
  aitken:
    D_min: 0.0087
    D_max: 0.052
    sigma: 1.6
  accumulation:
    D_min: 0.0535
    D_max: 0.44
    sigma: 1.8
  coarse:
    D_min: 1.0
    D_max: 4.0
    sigma: 1.8
  primary_carbon:
    D_min: 0.01
    D_max: 0.1
    sigma: 1.6
\end{verbatim}

Particle diameters and $\sigma$ are measured in $\mu\mathrm{m}$.

The \verb modes  section is essentially a map whose keys are mode names
(\verb aitken , \verb accumulation , \verb coarse , and \verb primary_carbon ,
in the example above), and whose values are themselves maps. The map for a given
mode contains the following fields:

\begin{itemize}
  \item \verb D_min : the minimum diameter of particles belonging to the mode
  \item \verb D_max : the maximum diameter of particles belonging to the mode
  \item \verb sigma : the geometric mean standard deviation for the size
                      distribution for particles belonging to the mode
\end{itemize}

Everything in a mode must be completely specified---there are no default values.

\subsubsection*{Aerosol Species}

Aerosol species populate each of the defined modes, and are defined in the
\verb aerosols  section. As we discussed in \refsection{modes_and_species}, a
species is defined by its elemental composition and its electric charge (given
in units of the electronic charge $|e|$).
%The \verb species  section follows the same format as defined by
%\href{https://cantera.org/documentation/dev/sphinx/html/yaml/species.html}{Cantera}.

Like the \verb modes  section, the \verb aerosols  section is a map whose keys
are {\em symbolic names} of aerosol particle species (e.g. \verb SO4  for
sulfate particles):

\begin{verbatim}
aerosols:
  SO4:
    name: sulfate
%    composition: {S: 1, O: 4}
\end{verbatim}

An aerosol species has the following fields:

\begin{itemize}
  \item \verb name : the full name for the species (e.g. \verb sulfate  for
    \verb SO4 ). You don't need to quote the full name, even if
                     it contains spaces.
\end{itemize}

\subsubsection*{Gas Species}

Gas species exist in the atmosphere, and aren't associated with modes. These
species are defined in the \verb gases  section.

The \verb gases  section is identical in structure to the
\verb aerosols section:

\begin{verbatim}
gases:
  SO2:
    name: sulfur dioxide
\end{verbatim}

A gas species has the following fields:

\begin{itemize}
  \item \verb name : the full name for a gas species (e.g.
                     {\verb sulfur dioxide} for \verb SO4 . You don't need to quote the
                     full name, even if it contains spaces.
\end{itemize}

\subsubsection*{Physics}

In this section, we tell the Haero driver what physical processes we wish
to simulate. Every field in this section assumes a \verb true  or
\verb false  value, so it's really just a set of ON/OFF switches. Valid fields
are:

\begin{itemize}
  \item \verb growth : enables particle growth modeling
  \item \verb gas_chemistry : enables the gas chemistry mechanism
  \item \verb cloud_chemistry : enables the cloud chemistry mechanism
  \item \verb gas_aerosol_exchange : enables exchange processes that occur
                                     between gas and aerosol particles
  \item \verb mode_transfer : enables the transferring particles between modes
  \item \verb nucleation : enables aerosol particle formation processes
  \item \verb coagulation : enables inelastic aerosol collision processes
\end{itemize}

\subsubsection*{Atmospheric Conditions}

In each cell within each column, there exist atmospheric conditions that
provide important parameters---temperature, pressure, relative humidity, etc---
that govern physical processes. The \verb atmosphere  section allows you
to specify one of a handful of simple models for obtaining those parameters.
Each model has its own parameters.

First and foremost, you specify a model with the \verb model  field within
the \verb atmosphere  section. Supported models are:

\begin{itemize}
  \item \verb uniform : a simple atmospheric environment in which columns are
                        assumed to be short in comparison to the height of the
                        atmosphere so that all conditions are uniform
  \item \verb hydrostatic : an atmospheric environment in hydrostatic
                            equilibrium, with the relationship between pressure
                            and temperature defined by an ideal gas law
\end{itemize}

These models are described in detail in \refsection{contexts}. Here we simply
list valid fields for each model. These fields are specified alongside the
\verb model  field in the \verb atmosphere  section. Tables~\ref{tab:uniform_atm}
and~\ref{tab:hydrostatic_atm} list fields for the \verb uniform  and
\verb hydrostatic  models.

\begin{table}[htbp]
\caption{Uniform atmosphere parameters}
\centering
\label{tab:uniform_atm}
\begin{tabular}{ccc}
  \toprule
  Parameter   & Description                  & Units   \\
  \midrule
  \verb mu    & Mean molecular weight of air & kg/mol  \\
  \verb H     & Scaled atmospheric height    & m       \\
  \verb p0    & Pressure                     & Pa      \\
  \verb T0    & Temperature                  & K       \\
  \verb phi0  & Relative humidity            & -       \\
  \verb N0    & Cloud fraction               & -       \\
  \bottomrule
\end{tabular}
\end{table}

\begin{table}[htbp]
\caption{Hydrostatic atmosphere parameters}
\centering
\label{tab:hydrostatic_atm}
\begin{tabular}{ccc}
  \toprule
  Parameter   & Description                  & Units   \\
  \midrule
  \verb mu    & Mean molecular weight of air & kg/mol  \\
  \verb H     & Scaled atmospheric height    & m       \\
  \verb p0    & Pressure                     & Pa      \\
  \verb T0    & Temperature                  & K       \\
  \verb phi0  & Relative humidity            & -       \\
  \verb N0    & Cloud fraction               & -       \\
  \bottomrule
\end{tabular}
\end{table}

\subsubsection*{Grid Parameters}

\subsubsection*{Initial Conditions}

The \verb initial_conditions section defines the initial state of an aerosol
system. These include number concentrations for species.

\begin{verbatim}
initial-conditions:
  SO2:
    concentration: 0.01
\end{verbatim}

This section is a mapping from species names to their initial concentrations.

\subsubsection*{Chemistry Model}

\subsubsection*{Simulation Parameters}

\subsection{Examples}

You can find examples of driver input files in the \verb driver/tests  directory
within the \verb haero  GitHub repository.
